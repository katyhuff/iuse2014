\documentclass{proposalnsf}
% This class file has been tweaked to death by LBarba to fit precisely the 
% formatting strictures of NSF, while still being rather pretty.

%%--------------------------------------------------------------------  PROCESS WITH XeLaTeX
%\usepackage{fontspec}% provides font selecting commands 
%\usepackage{paralist}       % compactitem environment
%\usepackage{xunicode}% provides unicode character macros 
%\usepackage{xltxtra} % provides some fixes/extras 
%\setromanfont[Mapping=tex-text,
%                 SmallCapsFont={Palatino},
%                 SmallCapsFeatures={Scale=0.85}]{Palatino}
%\setsansfont[Scale=0.85]{Trebuchet MS} 
%\setmonofont[Scale=0.85]{Monaco}

\renewcommand{\captionlabelfont}{\bf\sffamily}
\usepackage[hang,flushmargin]{footmisc} 
% 'hang' flushes the footnote marker to the left,  'flushmargin'  flushes the text as well.

% Define the color to use in links:
\definecolor{linkcol}{rgb}{0.459,0.071,0.294}
\definecolor{sectcol}{rgb}{0.63,0.16,0.16} % {0,0,0}
\definecolor{propcol}{rgb}{0.75,0.0,0.04}

\definecolor{gray}{rgb}{0.25,0.25,0.25}
\definecolor{ngreen}{rgb}{0.7,0.7,0.7} % a darker shade of green

\usepackage[
    %xetex,
    pdftitle={NSF proposal},
    pdfauthor={Rachel Slaybaugh, Kaitlin Thaney, Lorena Barba, C. Titus Brown, 
    Paul Wilson, Ethan White, Tracy Teal, Greg Wilson, and Kathryn Huff},
    pdfpagemode={UseOutlines},
    pdfpagelayout={TwoColumnRight},
    bookmarks, bookmarksopen,bookmarksnumbered={True},
    pdfstartview={FitH},
    colorlinks, linkcolor={sectcol},citecolor={sectcol},urlcolor={sectcol}
    ]{hyperref}

%% Define a new style for the url package that will use a smaller font.
\makeatletter
\def\url@leostyle{%
  \@ifundefined{selectfont}{\def\UrlFont{\sf}}{\def\UrlFont{\small\ttfamily}}}
\makeatother
%% Now actually use the newly defined style.
\urlstyle{leo}


% this handles hanging indents for publications
\def\rrr#1\\{\par
\medskip\hbox{\vbox{\parindent=2em\hsize=6.12in
\hangindent=4em\hangafter=1#1}}}

\addto\captionsamerican{%
  \renewcommand{\refname}%
    {References Cited}%
} % solution found here: http://www.tex.ac.uk/cgi-bin/texfaq2html?label=latexwords

\def\baselinestretch{1}
\setlength{\parindent}{0mm} \setlength{\parskip}{0.8em}

\newlength{\up}
\setlength{\up}{-4mm}

\newlength{\hup}
\setlength{\hup}{-2mm}

\sectionfont{\large\bfseries\color{sectcol}\vspace{-2mm}}
\subsectionfont{\normalsize\it\bfseries\vspace{-4mm}}
\subsubsectionfont{\normalsize\mdseries\itshape\vspace{-4mm}} %\itshape
\paragraphfont{\bfseries}

% ---------------------------------------------------------------------
% DRAFTING COMMENTS:
\newcommand\ignore[1]{} % Styles for author comments:
% Enter a comment like this:   \comment{This is a comment.}
%\ignore{
\newcommand{\important}[1]{\textcolor{red}{ #1 }}
\newcommand{\comment}[1]{\textcolor{blue}{ #1 }}
%}
% Uncomment lines below to change from visible to invisible comments.
%\renewcommand{\important}[1]{}
%\renewcommand{\comment}[1]{}
\usepackage{tabu}

\begin{document}

\newpage

% ----------------------  TECHNICAL PROPOSAL starts here ... Maximum 15 pages
\pagestyle{plain}
% reset page numbering to 1 
\pagenumbering{arabic}
\renewcommand{\thepage} {\arabic{page}}

\begin{center}
\small{PROJECT DESCRIPTION}

% Small title with normal-size font

\large\textbf{The Impact of Intensive Software Skills Training on Students' Scientific Careers}

{\large \sf Rachel Slaybaugh, Kaitlin Thaney, Lorena Barba, C.\ Titus Brown, and Paul Wilson\\ 
  with Ethan White, Tracy Teal, Greg Wilson, and Kathryn Huff}

\end{center}

\section{The Problem: Elusive Computational Competence in Science}

Scientists and engineers invented electronic computers to accelerate
their work, but two generations later, many researchers in science,
technology, engineering, and mathematics (STEM) are still not
\emph{computationally competent}: they do repetitive tasks manually
instead of automating them, develop software using a methodology best
summarized as ``copy, paste, tweak, and pray'' and fail to track
their work in any systematic, reproducible way.

While the World-Wide Web was created by a scientist to help his
peers share information, many still use it primarily as a way to find
and download PDFs.  Researchers may understand that open data can fuel
new insights but often lack the skills needed to create and provide a
reusable data set.  Equally, any discussion of changing scientific
publishing, making research reproducible, or using the web to support
``science as a service'' must eventually address the lack of
pre-requisite skills in the general STEM research community.

Studies have repeatedly shown that most researchers learn what they
know about computing by word of mouth \cite{hannay2009}, but this
approach is failing to meet present-day needs: most faculty would
agree that today's graduates are no more able to use computing and the
web \emph{in their research} than they were a generation ago.
Attempts to integrate more training in basic computing skills into
undergraduate education have largely failed to take root for several
reasons:

\begin{enumerate}

\item
  \emph{The curriculum is full.}  Undergraduate STEM programs already
  struggle to cover material regarded as core to their field.  While
  many scientists would agree that more material on programming,
  reproducible research, or web-enabled science would be useful, there
  is no consensus on what to take out to make room.

\item
  \emph{The blind leading the blind.}  Many faculty lack computational
  skills themselves and, hence, are unable to pass them on.

\item
  \emph{Cultural difference.}  Scientists and software developers have
  different priorities and different approaches to problem solving,
  which often impedes collaboration and knowledge transfer
  \cite{segal2005a}.

\end{enumerate}

One final issue is that \emph{the rewards are unknown}.  Open,
web-based science is still in its infancy, so there is no general
understanding of what people might need to know in order to
incorporate it into their research careers.  Since it is hard to
measure something if you don't know what to look for, or if it is so
young that there hasn't actually \emph{been} long-term impact, little
systematic study has been done to date of whether early training in
the skills needed for this new kind of science actually has an impact,
and if so, how and how much.  Without such feedback, there is no
systematic way to improve the training programs that currently exist.

\section{Proposed Work: Leveraging Proven Curriculum to Promote Computational Competence in Science}

This proposal builds on the success to date of the Software Carpentry
workshops (Section~\ref{sec:SC}), a proven curriculum of essential
software skills that enhance the productivity of graduate students,
post-docs, and faculty.  We propose to:

\begin{enumerate}

\item
  conduct formative evaluation of the impact of software skills
  training for undergraduates likely to continue in research careers
  as they progress through the early stages of those careers;

\item
  conduct summative evaluation of the training's overall impact on a
  multi-year timescale in order to improve the content and
  presentation of the training; and

\item
  disseminate the resulting curriculum widely.

\end{enumerate}

We will run software-skills workshops for undergraduate students
taking part each year in summer research opportunities such as the
NSF's Research Experience for Undergraduates (REU) program, at or near
the start of those students' time in the lab.  Based on data already
collected from Software Carpentry workshops, we expect this training
will help them be more productive during their research (graduate-level
participants in our existing workshops typically report that what we
teach saves them a day per week) and will prepare them to work in a
world where all aspects of science are increasingly dependent on
computing.

These undergraduates will serve as the treatment population for a
five-year study of the impact of this training on their careers in
general, and their involvement with open and web-enabled science in
particular.  In order to conduct this study, we will hire an expert in
educational assessment, whose full-time work for the duration of the
project will be to explore the effects of the training on workshop
participants.

\subsection{Workshops: A Distributed Model for National Impact}

We will run two-day workshops at a steadily increasing number of sites
each year for five years, timed to coincide with the start of the
summer influx of undergraduate research students.  Each workshop will
be offered to a minimum of 40 learners per site, giving us a target
study population of at least 2200 students by year 5.  The content will be
tailored to meet local needs but will be based on what is being used
at that time by Software Carpentry and affiliated educational efforts.
By design, it is straightforward to adapt workshop materials and
contribute changes to the Software Carpentry course material.  These
features enhance the portability and flexibility of the workshops and
increase the likelihood of wide dissemination beyond this project.

The home sites for investigators named in this proposal (George
Washington University, Michigan State University, University of
California, Berkeley, University of Wisconsin -- Madison, and
Utah State University) will run workshops in each of those years.
Other sites will be added each year, expanding the total to 15 in
year 5 and accordingly increasing the size of our study population.  We will
focus expansion on NSF REU sites \cite{nsfreu} but, as detailed in
Section~\ref{sec:diversity}, we will also offer some workshops to
other communities.

One set of possible sites for expansion are those campuses included in
the ``Condo of Condos'' consortium, recently recommended for funding
by the National Science Foundation.  The Software Carpentry workshops
proposed here will be valuable to that consortium in meeting its goals
of increasing the number and diversity of researchers using advanced
cyber infrastructure and of developing data science practitioners.

Beyond this consortium, we will recruit sites for hosting workshops by identifying locations at which we could have the largest impact and/or that contribute most to our goal of increasing diversity. If we find there are more sites interested in hosting workshops than we are able to support in a given year, we will select the subset of sites that best meet our goals.

\subsection{Curriculum: From Tools to Techniques to Concepts}
\label{sec:workshops}

While there is considerable scope for customizing workshops to
accommodate learners' prior experience and discipline-specific needs,
what these workshops seek to convey is the best
practices a researcher needs to be \emph{computationally competent}
\cite{wilson2013}:

\begin{compactitem}
\item
  how to create, use, and share structured data
\item
  how to automate repetitive tasks; 
\item
  how to track and share work over the web; and
\item
  how to grow a program in a modular, testable, reusable way.
\end{compactitem}

With these objectives, the base workshop format will be divided in to four 
modules.  All workshops will be hands-on so that learners can ``learn by doing'' 
and have experience with concrete examples. A typical workshop devotes roughly 
half of a day to each of these four modules:

\begin{enumerate}

\item Working with Data - this module will begin with a spreadsheet program in which participants will learn how to organize, analyze, and export data.  Learners will work with data representative of their field of study.  By the end of this module, learners will be able to sort data, conduct calculations, and export data to more generic file types, such as tab-delimited.  This module sets the stage for learning to work with data outside of a spreadsheet for automation, data parsing, databases, data sharing, and more sophisticated statistical analyses.

\item Automation, the Unix Shell - this module will introduce learners to the shell; teach how to view, search, and manipulate text files at the command line; and introduce basic automation at the command line.

\item Automation, Introductory Programming - this module teaches learners 
  introductory Python or R.  Learners will be able to write short scripts and 
  work in the programming environments IPython Notebook or RStudio.  Learners will experience the potential of programming to automate analysis and data processing.  This module will also teach a way of thinking about computation, with loops, if-then, and for statements, that is essential for computational competency and a true understanding of the potential for computation in research.

\item Version Control and Data Sharing - this module teaches learners how to keep track of their data and analyses in an open and reproducible way.  The lessons will focus on data management strategies as well as provide an introduction to GitHub for version control of scripts, programs, analyses , etc.
\end{enumerate}
%
For context, Table~\ref{table:schedule} shows a sample workshop schedule.

\begin{table}
\caption{Sample Software Carpentry Workshop Schedule}
\begin{center}
\begin{tabu}[h]{l l l l}
\textbf{Day 1} & & & \\ \hline
9:00 &-& 9:30 & Introduction \\
9:30 &-& 10:45 & Working with Data - data manipulation in spreadsheets \\
10:45&-& 11:00 & coffee \\
11:00&-& 12:00 & Working with Data (cont.) - data manipulation in spreadsheets and data export \\
12:00&-& 1:00 & lunch \\
1:00 &-& 2:45 & The Unix Shell - fundamentals and text manipulation \\
2:45 &-& 3:00 & coffee \\
3:00 &-& 4:30 & The Unix Shell - scripts and automation (cont.) \\ 
& & & \\
\textbf{Day 2} & & & \\ \hline
9:00 &-& 10:30 & Introductory Programming - the programming environment and data types \\
10:30&-& 10:45 & coffee \\
10:45&-& 12:00 & Introductory Programming (cont.) - automation and analysis with scripts  \\
12:00&-& 1:00 & lunch \\
1:00 &-& 2:45 & Version Control with Git - keeping track of versions of scripts  \\
2:45 &-& 3:00 & coffee \\
3:00 &-& 4:30 & Data Management and Reproducible Research
\end{tabu}
\end{center}
\label{table:schedule}
\end{table}

As the module descriptions suggest, our real aim isn't to
teach Python, Git, or any other specific tool: it's to teach
\emph{computational competence}. We can't do this in the abstract:
people won't show up for a hand-waving talk and, even if they do, they
won't understand. If we show them how to solve a specific problem with
a specific tool, though, we can then lead into a larger discussion of
how scientists ought to develop, use, and curate software.

These workshops strive to show people how the pieces fit together: how to write a
Python script that fits into a Unix pipeline, how to automate unit
tests, etc. Doing this gives us a chance to reinforce ideas and also
increases the odds of participants being able to apply what they've learned
once the workshop is over.

\subsection{Execution: Quality Instruction}
We will aim for no more than 40 people per room at a workshop, so that
every learner can receive personal attention when needed.  Where
possible, we will run two or more rooms side by side and use a
pre-assessment questionnaire (see Section~\ref{sec:questionnaire} for an excerpt 
of currently-used questions) as a sorting hat to group learners by
prior experience, which simplifies teaching and improves their
experience.  %We will not shuffle people from one room to
%another between the first and second day: with the best
%inter-instructor coordination in the world, it still results in
%duplication, missed topics, and jokes that make no sense.

All of the workshop instructors will have been trained and certified
by Software Carpentry and will have had experience teaching this
material prior to engaging in these particular workshops.  Just as
importantly, all instructors will themselves be working scientists. By virtue of 
using  these skills and concepts daily in their own research they are better 
able both to serve as role models and to deal with unanticipated
questions or challenges based on personal experience (e.g.,
\cite{ram2013}). 

Software Carpentry has a rich network of trained instructors, often recruited 
from past workshops. These instructors are volunteers who participate for a 
variety of reasons including sharpening their own teaching and computing skills, 
increasing diversity in the pipeline, and because it's fun. 

As well as instructors, we will rely on local helpers to wander the room and
answer questions during practicals. These helpers may be participants in
previous workshops who are interested in becoming instructors, graduate
students who've picked up some or all of this on their own, or members
of the local open source community; where possible, we will aim to have at
least one helper for every eight learners.

\subsection{Increasing Diversity: Changing the Odds for Underrepresented Scientists}
\label{sec:diversity}

In order to increase the diversity of the study population, at least
one workshop in each year will be aimed specifically at female
students.  Software Carpentry's first such workshop, held in Boston in
June 2013, attracted 120 participants; its second is scheduled for
Lawrence Berkeley National Laboratory in April 2014, and at least two
more will be held by the time work on this project commences (one in
the United States and one in Europe).  This work will build on that
experience and draw on the pool of instructors who have gained
mentoring experience through those specific workshops.

Finally, we will organize workshops in years 2 through 5 specifically aimed at students from minority serving institutions.
We are already in contact with the Computing Alliance for
Hispanic-Serving Institutions (CAHSI) and with the Association of
Public and Land-grant Universities' program for historically black
colleges and universities (HBCUs). Software Carpentry is running its
first workshop at an HBCU (Spelman College) in early 2014, and we expect to
have significantly expanded these efforts by year two of this project.

%We have found workshops are more effective if people come in groups (e.g., 4--5
%people from one lab) or have other pre-existing ties (e.g., the same
%disciplinary background). They are less inhibited about asking
%questions, and can support each other (morally and technically) when
%the time comes to put what they've learned into practice after the
%workshop is over. This is one reason why structuring workshops around
%REU summers is likely to have a high impact.
%Group signups also yield much higher turnout from
%groups that are otherwise often under-represented, such as women and
%minority students, since they know in advance that they will be in a
%supportive environment.

\subsection{Formative and Summative Assessment: Maximizing Learning and Impact}

We will employ an expert in educational assessment full-time for five
years to monitor and compare undergraduate participants in these
software carpentry skills building workshops, participants in a subset
of our regular (graduate-level) workshops, and non-participants (as a
control population).  As part of their work, this person will be
responsible not only for collecting and analyzing data but also for
refining and extending the methods and measures used to gauge impact. D-Lab (Section \ref{sec:dlab}) will assist in locating and supporting this expert.

Assessment will build on previous work (see Section~\ref{sec:pastAssessment}), focusing particularly, but not exclusively, on the following questions:

\begin{enumerate}
\item
  Are students who receive this training more likely than their peers
  to develop new tools and practices and/or become involved in
  outreach and education activities (i.e., are they more likely to
  become creators and leaders)?

\item
  Are students who receive this training more likely than their peers
  to incorporate open science and/or web-enabled science tools and
  practices into their work?

\item
  Do outcomes differ between women and underrepresented minorities on
  one hand and non-underrepresented minorities and men on the other?
  If so, in what ways, and what steps are effective in correcting for
  these differences?

\item
  In what ways does this training change students' outlook on the
  practice of science itself?

\item
  Are students who receive this training more likely than their peers
  to choose computationally oriented research topics and/or careers?
  Are those who do not choose computationally oriented paths
  nevertheless more likely to incorporate the tools and practices
  mentioned above into their work?

\item
  Are students who receive this training more likely to continue to
  graduate school than their peers?

\end{enumerate}

This expert in educational assessment will explore ways in which our engagement with
students changes the outlook and work practices of their peers and
faculty supervisors (i.e., whether there is knowledge transfer
sideways and upward) and will explore the effectiveness of the community
building and dissemination activities detailed in the next sections. 

As with Software Carpentry's work to date,
assessment will use both qualitative and quantitative techniques.  On
the qualitative side, we will conduct a series of interviews over the
five-year period of the study to see how attitudes, aspirations, and
activities change. Quantitatively, we will measure uptake of key
tools such as version control as a proxy for adoption of related
practices, as well as exploring more traditional measures of research
success, such as progression to graduate school and
publication/citation rates. 

To track impact over time, we will conduct pre-workshop surveys 
and interviews roughly in the week before the workshops and the
series of post-workshop surveys and interviews beginning approximately 
one month following the workshop. Follow on surveys and interviews
will be conducted and the end of the students' lab appointment 
and annually thereafter (as applicable based on year of participation
in the project). Sections~\ref{sec:survey} and ~\ref{sec:interview} provide some example survey and interview questions, respectively.

Our findings, and any new methods or measures developed, will be
shared with other researchers through publication in peer-reviewed
journals and high-profile conferences (Section~\ref{sec:pub}).

\subsection{Community Building: Supporting Computational Competence}
\label{sec:pub}

We will employ one graduate student part-time at each site named in this proposal each year to provide technical support to workshop participants,
and to act as an anchor for a Hacker Within-style grassroots group at
that site (Section~\ref{sec:THW}).  These community liaisons will not
be study subjects but will help us stay in touch with students who
are (a key requirement for any longitudinal study).

ANOTHER PARAGRAPH ABOUT WHAT THIS WILL ACTUALLY LOOK LIKE.

Separately, the Mozilla Science Lab will focus part of its ongoing
community engagement efforts on the students who have taken part in
our workshops during both the remainder of their undergraduate careers
and afterward to help them become part of the broader
open science community.  This may include helping the students
organize and run workshops of their own in subsequent years,
connecting them with other open science projects, introducing them to
potential graduate supervisors who understand and value their new
skills and outlook, etc.

As a subordinate part of their work, the researcher employed by this
project will assess the effectiveness of the local graduate student
organizers.  In particular, they will explore whether seeding activity
in this way leads to the formation of self-sustaining grassroots
groups, and if so, what activities those groups develop on their own,
how (and how effectively) they share discoveries with each other, the
extent to which alumni of this program stay engaged with these groups,
and whether the presence of these groups has a demonstrable impact on
students' career paths in general and/or on their engagement with
open and web-enabled science in particular.

While it will not be feasible to bring all of the students
participating in a given year's workshops together physically, we will
organize and run virtual conferences toward the end of their research
term to give them an opportunity to present their work to one another,
discuss what they have learned and build peer-to-peer connections.
These conferences will also provide an opportunity to introduce
participants to new forms of scientific ``publishing'', including
blogging, the creation of screencasts and demonstration videos and
other methods that may not yet exist.

\subsection{Curriculum Development and Dissemination: Expanding the Impact}

We will employ one instructional designer part-time throughout the
project to create new material and to improve existing material based
on feedback from workshop participants and the assessment program.
Here, ``creating material'' may include both designing and
implementing new domain-specific learning modules and translating
existing materials into new forms, such as video recordings of
lectures or auto-graded quizzes for self-paced instruction.  This work
will be done in consultation with educators at participating
institutions in order to encourage incorporation of those materials
into existing curricula.

All of the materials produced by and for this project will be made
freely available under the Creative Commons -- Attribution (CC-BY)
license.  The instructional designer will work with the Mozilla
Science Lab and affiliated groups to share these materials and the
results of our studies of the program's impact, through science
education journals, conferences, and other channels. Specifically, we
plan to publish articles in:

\begin{compactitem}

\item
  \emph{Science Education}

\item
  \emph{Physics Education}

\item
  \emph{Journal of Computers in Mathematics and Science Teaching}

\item
  \emph{American Scientist}

\end{compactitem}

and at the following conferences:

\begin{compactitem}

\item
  the Special Interest Group on Computer Science Education (SIGCSE) of
  the Association for Computing Machinery, Inc. (ACM)

\item
  the International Conference on Physics Education (ICPE)

\item
  the Association for Biology Laboratory Education (ABLE)

\item
  the Society for Industrial and Applied Mathematics (SIAM)'s
  education session at the Joint Mathematics Meeting

\item
  the Education Training and Workforce Development Division's track at
  the American Nuclear Society (ANS)'s annual meeting

\end{compactitem}

The dissemination of this project's curriculum has strong potential to
be high. Current Software Carpentry materials are available as online lessons and in GitHub. Workshop materials will continue to be open access and flexible, thus they can be readily adopted by others. Adapting workshop materials is
low cost and does not require special equipment. Workshop materials
are structured such that they can scale to the size and application of
interest to a particular group. Anyone using
workshop materials can directly contribute changes and feedback, which
both increases buy-in and improves the materials organically. The Software Carpentry infrastructure
provides support in the form of materials and people. And
finally, local chapters of The Hacker Within create a natural
ecosystem of support for workshop participants, their peers, and
faculty.

As a subordinate part of their work, the researcher employed by this
project will assess the extent to which curriculum developed during
this program is taken up by other educators (particularly those who
think of themselves as scientists first and computationalists second),
and their perception of its utility.  Mid-point results of this
evaluation will be shared with the instructional designer in order to
allow evidence-based improvement of the materials.

\subsection{Project Management}

Prof.\ Slaybaugh will be responsible for overall project management
and reporting.  The educational assessment expert hired by this
project will report directly to her.  Prof.\ Slaybaugh will be
assisted by Dr.\ Huff, who will manage and coordinate the graduate
student assistants at each site.  Dr.\ Huff will also be responsible
for organizing the workshops aimed at female students.

Profs.\ Barba, Brown, White, and P.\ Wilson will be responsible for
coordinating workshops and for recruiting and supervising the graduate
student assistant at their respective institutions. Prof.\ Teal will co-coordinate the workshops held at Michigan State University and will assist in developing workshop materials.

Dr.\ G.\ Wilson and the half-time instructional designer hired by this
project will be responsible for preparation and publication of
learning materials.  Dr.\ G.\ Wilson will also provide instructional
training for the graduate student assistants and other participants in
the project on an ongoing basis and will be responsible for
organizing the workshops aimed at students from minority serving institutions.

Workshop operations (such as finding instructors and arranging their
travel) will be handled by Mozilla staff who are performing these
duties for the Software Carpentry program more generally.  These staff
will be supervised by Ms.\ Thaney, who will also be responsible for
connecting the other PIs and the graduate student assistants with
other open and web-enabled science groups.

\section{Related Work}

\subsection{Theoretical Positioning}

Our theory of action is straightforward: if students are explicitly
taught software skills in a way that makes them seem both useful and
important, they are likely to begin using them in day-to-day work,
which will create a positive feedback cycle leading them to acquire
more (and more advanced) skills on their own. This positive feedback
cycle will in turn result in the students being more likely to engage
in open and web-enabled scientific activities that would otherwise
have been unknown, incomprehensible, or out of reach.

Using the terminology of \cite{guidelines}, our work is primarily
\emph{design and development research}. We plan to design and develop
solutions related to student engagement and mastery of specific
skills, drawing on existing evidence from Software Carpentry's
graduate-level workshops and investigating their impact and
effectiveness. Further, we further plan to design and iteratively
develop interventions. We are ready to begin collecting data on the
feasibility of implementing solutions in typical delivery settings.

\subsection{Research}

Studies of how scientists use computers and the web have found that
most scientists learn what they know about developing software and
using computers and the web in their research haphazardly and through
word of mouth \cite{hannay2009,prabhu2011}. In our experience, most
training meant to address this issue:

\begin{compactitem}

\item
  does not target scientists' specific needs (e.g., is a general
  ``Introduction to Computing'' class shared with students majoring in
  other areas);

\item
  only covers the mechanics of programming in a particular language
  rather than giving a complete picture, including data management,
  web-enabled publishing, the ``defense in depth'' approach to
  correctness discussed in \cite{dubois2005}, or the other
  foundational skills laid out in \cite{wilson2013}; and/or

\item
  jumps to advanced topics such as parallel computing before
  scientists have mastered the foundations.  Most research on
  scientific computing, such as \cite{hochstein2005}, does the same.

\end{compactitem}

On the other hand, studies of how people in general learn to program,
and of how effective different approaches to teaching them are, have
made significant strides in the past decade.  In particular, our work
is informed by the long-running research program of Guzdial et al.\ at
Georgia Tech, who have found that a ``media first'' introduction to
computing outperforms more conventional alternatives
\cite{guzdial2013} and that it is possible to assess the extent to
which programming concepts, rather than merely the syntax of a
particular programming language, have been mastered \cite{tew2011}.

Others (e.g., \cite{porter2013}) have demonstrated that peer
instruction is a significantly better way to teach introductory
programming than conventional classroom approaches.  As discussed in
the section below, we are already working to incorporate these
evidence-based approaches into our teaching and will accelerate these
efforts within the scope of this award.

\subsection{Software Carpentry}
\label{sec:SC}

Software Carpentry \cite{swcsite,wilson2012} is the largest effort to
date to address issues surrounding inadequate software carpentry skill
training for students. Originally created as a training program at Los
Alamos National Laboratory in the late 1990s, it is now part of the
Mozilla Science Lab's efforts to help scientists take advantage of
ways in which the web can change the practice of science today and
invent new ways tomorrow.  Over 100 certified volunteer instructors
delivered two-day intensive workshops like those described in Section~\ref{sec:workshops} to more than 4200 people in 2013
alone.

Software Carpentry's curriculum and teaching practices have been
refined via iterative design and are informed by current research on
teaching and learning best practices.  


Its instructor-training program \cite{trainingsite}, which takes 2--4 hours/week of a
trainee's time for 12--14 weeks (depending on scheduling interruptions), introduces participants to a variety of modern
teaching techniques (e.g., peer instruction, active learning, and
understanding by design), to concepts underlying these techniques
(e.g., cognitive load theory), and to topic-specific work by computing
education researchers (see \cite{guzdial2010}, \cite{hazzan2011}, and
the first third of \cite{sorva2012} for overviews).  One example of
how they translate theory into practice is their insistence on live
coding during teaching as a way of demonstrating and transferring
authentic practice to learners.

Evidence of instructor training and experience is based on a Mozilla's
Open Badge program \cite{badges2014}. Further levels of expertise are
obtained through helping at a workshop, being an instructor, being a 
lead instructor, and developing workshop materials. All of the 
participants in this proposal are certified instructors and have 
taught or will teach Software Carpentry workshops prior to the start
of this project. For example, Dr.\ Huff has already taught at 9 
workshops and Dr.\ G.\ Wilson has served as an instructor at 35. 

\subsubsection{Past Assessment}
\label{sec:pastAssessment}

Software Carpentry has been assessing learning outcomes and retention
since the beginning of its Sloan Foundation funding in January 2012.
The first round of assessment included both qualitative and
quantitative assessment by Dr.\ Jorge Aranda (then at the University
of Victoria) and Prof.\ Julie Libarkin (Michigan State University).

Dr.\ Aranda surveyed and interviewed participants, observed a
workshop, and analyzed screencasts of participants working through a
programming assignment. The surveys and interviews were conducted 
both pre- and post-workshop. The survey asked questions regarding
the software development habits of respondents, the tools they were familiar with, their level of knowledge of five core Software
Carpentry topics (shell commands, Python, version control, SQL, and 
testing concepts), and their challenges in using scientific computing
to answer their research questions. 

According to both qualitative and quantitative data,
Dr.\ Aranda found significant increases in
participants' understanding and use of shell commands, version control
tools, Python, and testing techniques. Perhaps more importantly,
participants reported better proficiency with software tools; greater
concern for issues of provenance and code quality; better strategies
to approach software development; and \emph{new research questions} that have
become accessible thanks to an increase in participants' software
development skills.

Students also took a ``quiz" consisted of Yes/No questions, that were purposefully chosen so that only about half of them could be answered with the standard material in the workshop; the other half was not covered by workshop instruction. Additionally, a cross-cutting half could be answered with introductory familiarity to the topic in question, while the other half would represent more advanced levels of expertise. %In other words, the quiz was designed so that there were an easy and a hard question answerable through workshop instruction, and an easy and a hard question not answerable through workshop instruction. 

The objectives for this were to assess whether participants not only learned the workshop materials but were exploring the topics in greater depth on their own, and to avoid ceiling effects in our survey. Quiz performance improved across the board (by $\sim$30\%) for all question categories.

Prof.\ Libarkin performed a more detailed assessment of participants
in a workshop held at Michigan State University, which was attended remotely by students from
the University of Texas at Austin. Prof.\ Libarkin also collected qualitative and quantitative data. Eighty-five percent of participants
reported that they learned what they hoped to learn, 81\% changed
their computational understanding, and 96\% said they would recommend
the workshop to others. 

An attempt to scale this up in 2013 was set back by personnel changes,
but systematic follow-ups with past participants in workshops have now
been resumed, and we expect to be able to present preliminary results
by mid-2014.

Some excerpted survey questions from the discussed studies
can be found in Section~\ref{sec:survey}, and excerpted interview 
questions from Dr.\ Aranda's study can be found in Section~\ref{sec:interview}. The studies mentioned here and the additional data that will become available within the next six months will serve as a starting point for this project's assessment expert. 

\subsection{The Hacker Within}
\label{sec:THW}

The Hacker Within (THW) was founded by graduate students, including Dr.\ Huff and Prof.\ Slaybaugh, in nuclear
engineering at the University of Wisconsin -- Madison to provide a
forum for sharing scientific computing skills and best practices with
their peers \cite{huff2011}. As THW matured as a student organization,
it attracted students from many scientific disciplines and academic
levels. THW conducted bi-weekly seminars and developed a series of
short courses addressing the programming languages C++, Python, and
Fortran; best practices such as version control and test-driven code
development; and basic skills such as UNIX mobility. This curriculum
was delivered primarily as interactive short workshops on campuses and
during scientific conferences. Many previous founders of the Hacker
Within have since become instructors with Software Carpentry, and a
new generation of THW graduate students has begun to emerge in their
place. In 2013, new branches of THW were initiated at the University
of Melbourne and the University of California, Berkeley (under the direction of Dr.\ Huff and Prof.\ Slaybaugh).

\subsection{Condo of Condos}
\label{sec:CofC}

The ``Condo of Condos'' consortium, led by Clemson University and
including the University of Wisconsin -- Madison and four other
campuses during its pilot phase, has recently been recommended for
funding by the National Science Foundation.  The consortium's primary
task is to build a network of advanced cyber infrastructure research
and education facilitators (ACI-REFs), with goals that include
increasing the diversity of researchers using advanced
cyber infrastructure on each campus and developing data science
practitioners.  The Software Carpentry workshops being designed under
this proposal will serve those goals directly.  As an investigator on
both that project and this proposal, Prof.\ P.\ Wilson will engage the
network of ACI-REFs to share this curriculum with both the initial
consortium institutions and any institutions who are able to join in
the future.

\subsection{D-Lab}
\label{sec:dlab}

D-Lab, located at the University of California, Berkeley, has the mission to create cross-disciplinary resources for high-level training and support services for social science researchers campus-wide. It does this by adaptively building new forms of shared infrastructure, including consulting and workshops in research tools and methods, and fostering discovery and connections with the pools of expertise and offerings of Berkeley's departments and professional schools. As a new research unit at Berkeley, D-lab has been actively involved in teaching software tools. D-lab is engaged in evaluating students’ experiences with those trainings and adapting subsequent trainings in response to their evaluations. D-lab is an enthusiastic supporter of campus efforts to broaden this approach to teaching.

D-lab will provide space, outreach, and assistance in locating and supporting the assessment position outlined in this proposal. Prof.\ Slaybaugh will engage with D-lab to capitalize on their expertise at the intersection assessment and scientific education. 

\section{Broader Impact}

We believe this work will have significant impact in several areas
beyond directly improving the computational science skills of workshop
participants.

\begin{enumerate}

\item
  \emph{Enhance economic competitiveness.} Computing is no longer
  optional in any part of science: even scientists who don't think of
  themselves as doing computational work rely on computers to prepare
  papers, store data, and collaborate with colleagues.  The better
  their computing skills are, the better prepared they will be to
  contribute to the research that underpins the nation's economic
  competitiveness.

\item
  \emph{Improving STEM education for everyone, not just participants.}
  By creating and validating high-quality open access teaching
  materials, and the methods used to deliver them, this project will
  enable improvement in STEM education for everyone, everywhere, not
  just for participating students and participating institutions.

\item
  \emph{Improving STEM education tomorrow, not just today.}  As noted
  in the introduction of this proposal, most of today's efforts to
  transfer computational skills to STEM researchers and connect them
  with 21st Century innovations in how science is done are flying
  blind: there is effectively no feedback from long-term impact to
  instructional action.  By creating and validating such a feedback
  loop---i.e., by showing scientists how to apply science to their
  teaching---this project will demonstrate how STEM education can be
  continuously improved.

\item
  \emph{Improve participation in STEM by women and underrepresented
    minorities.} The disproportionately low participation of women and
  some minority groups in STEM is well documented, as is the fact that
  computing is one of the least diverse fields within STEM.  This
  second fact creates a vicious cycle: people with weaker computing
  skills may be less competitive in research than their peers, which
  reduces their participation in activities viewed as non-core, which
  in turn results in them having weaker skills.  This project will
  strive to break this cycle by giving at-risk students an opportunity
  to ``level up'' in a supportive environment and by connecting them
  with mentors who can serve as role models.

\end{enumerate}

\section{Career Management Plan}

The graduate students who are serving as mentors for the
undergraduates at the different universities will each be paired with
a local faculty mentor. The faculty mentor will meet regularly with
the graduate student to discuss and problem solve any issues that the
graduate student or undergraduates are having and to provide active
mentoring on how to train students in computational approaches.

In addition to engaging with the graduate students on their mentoring
of the undergraduates, the faculty mentors will also serve as
mentors for computational aspects of the graduate students' research
and careers. In many areas of science, computationally-minded students
are located in labs where the PIs do not have strong computational
backgrounds. This means that they do not have a mentor to teach them
about good computational practice in research. In addition, they do
not have someone with whom to discuss computational careers, thus limiting
their exposure to career paths outside of academia. Because the
faculty mentors will have strong computational backgrounds themselves,
they can fill this void for computationally-minded students.

\section{Results from Prior NSF Support}

\subsection{Titus Brown}
\noindent {\bf Award number: NSF 09-23812,} Project title: Symbiont Separation and Investigation of the Novel Heterotrophic Osedax Symbiosis using Comparative Genomics;\\
{\it PI: Titus Brown}
\begin{compactitem}
\item[--] Total Award Amount: \$50,000
\item[--] Starting Date: 01/01/10; Ending Date: 12/31/13
\item[--] Summary of results: This project was a collaborative project with Dr.\ Shana Goffredi at Occidental College, where we analyzed sequence from MDA-amplified metagenomic samples taken from an Osedax bone eating worm. We produced the first high-quality assemblies of two symbiont genomes, giving us a whole-genome perspective on symbiont metabolism.
\item[--] Publications: ISME J in November 2013 (PubMed ID 24225886)
\end{compactitem}

\subsection{Lorena Barba}


\noindent {\bf  NSF Award No. OCI-1149784, }CAREER:  Scalable Algorithms for Extreme Computing on Heterogeneous  Hardware, with Applications in Fluids and Biology;\\  {\it PI: Lorena Barba}
\begin{compactitem}
\item[--] Total Award Amount: \$550,627
\item[--] Period of performance: 2/15/2012--2/14/2017
\item[--] Summary: The project investigates new algebraic applications of the fast multipole method (FMM) in elliptic PDE solvers and pre conditioners, and in fast matrix-vector products within iterative methods for solution of algebraic equations. In incompressible fluid dynamics, Poisson solvers take the majority of the compute time and are constrained in scalability by communication requirements. In biological applications, the project looks at boundary element method solutions of bioelectrostatics and Stokes flow problems, where a dense linear system results from the discretization. We have so far developed a new application of FMM, consisting of a relaxation of the multipole truncation number (i.e., lowering the FMM accuracy) as the iterations of a GMRES solver progress. This results in a four- to five-fold acceleration of the solution.
\item[--] Products: Comput. Phys. Comm., 184(3):445--455 (2013); Comput. \& Fluids, 80:17--27 (2013); SIAM News, 46(6):1 (July 2013); Comput. Phys. Comm., in press, online 4 Nov. 2013, doi:10.1016/j.cpc.2013.10.028.
\end{compactitem}

\section{Building Blocks: Excerpts from Assessment Tools}
\subsection{Pre-Assessment Questionnaire}
\label{sec:questionnaire}
The following is a subset of questions from the pre-assessment
questionnaire administered prior to Software Carpentry workshops. 
This questionnaire is used  to gauge participant skill level for so they can be
matched with an appropriate level of instruction.

\begin{itemize}
%\item
%  What is your discipline?
%  \begin{compactitem}
%    \item Space sciences
%    \item Physics
%    \item Chemistry
%    \item Earth sciences (geology, oceanography, meteorology)
%    \item Life science (ecology, zoology, botany)
%    \item Life science (biology, genetics)
%    \item Brain and neurosciences
%    \item Medicine
%    \item Engineering (civil, mechanical, chemical)
%    \item Computer science and electrical engineering
%    \item Economics
%    \item Humanities and social sciences
%    \item Tech support, lab tech, or support programmer
%    \item Administration
%    \item Other:
%  \end{compactitem}
%
%\item
%  In three sentences or less, please describe your current field of
%  work or your research question.
%
%\item
%  What OS will you use on the laptop you bring to the workshop?
%  \begin{compactitem}
%    \item Linux
%    \item Apple OS X
%    \item Windows
%    \item I do not know what operating system I use.
%  \end{compactitem}
%
%\item
%  With which programming languages, if any, could you write a program
%  from scratch that imports some data and calculates the mean and
%  standard deviation of that data?
%  \begin{compactitem}
%    \item C
%    \item C++
%    \item Perl
%    \item MATLAB
%    \item Python
%    \item R
%    \item Java
%    \item Other:
%  \end{compactitem}

\item
  What best describes how often you currently program?
  \begin{compactitem}
    \item I have never programmed.
    \item I program less than one a year.
    \item I program several times a year.
    \item I program once a month.
    \item I program once a week or more.
  \end{compactitem}

\item
  What best describes the complexity of your programming? (Choose all
  that apply.)
  \begin{compactitem}
    \item I have never programmed.
    \item I write scripts to analyze data.
    \item I write tools to use and that others can use.
    \item I am part of a team which develops software.
  \end{compactitem}

  \item
    A tab-delimited file has two columns showing the date and the
    highest temperature on that day. Write a program to produce a
    graph showing the average highest temperature for each month.
    \begin{compactitem}
    \item Could not complete.
    \item Could complete with documentation or search engine help.
    \item Could complete with little or no documentation or search engine help.
    \end{compactitem}

%  \item
%    How familiar are you with Git version control?
%    \begin{compactitem}
%    \item Not familiar with Git.
%    \item Only familiar with the name.
%    \item Familiar with Git but have never used it.
%    \item Familiar with Git because I have used or am using it.
%    \end{compactitem}

  \item
    Consider this task: given the URL for a project on GitHub, check
    out a working copy of that project, add a file called notes.txt,
    and commit the change.
    \begin{compactitem}
    \item Could not complete.
    \item Could complete with documentation or search engine help.
    \item Could complete with little or no documentation or search engine help.
    \end{compactitem}

%  \item
%    How familiar are you with unit testing and code coverage?
%    \begin{compactitem}
%    \item Not familiar with unit testing or code coverage.
%    \item Only familiar with the terms.
%    \item Familiar with unit testing or code coverage but have never used it.
%    \item Familiar with unit testing or code coverage because I have used or am using them.
%    \end{compactitem}
%
%  \item
%    Consider this task: given a 200-line function to test, write half
%    a dozen tests using a unit testing framework and use code coverage
%    to check that they exercise every line of the function.
%    \begin{compactitem}
%    \item Could not complete.
%    \item Could complete with documentation or search engine help.
%    \item Could complete with little or no documentation or search engine help.
%    \end{compactitem}
%
%  \item
%    How familiar are you with SQL?
%    \begin{compactitem}
%    \item Not familiar with SQL.
%    \item Only familiar with the name.
%    \item Familiar with SQL but have never used it.
%    \item Familiar with SQL because I have used or am using them.
%    \end{compactitem}
%
%  \item
%    Consider this task: a database has two tables: Scientist and
%    Lab. Scientist's columns are the scientist's user ID, name, and
%    email address; Lab's columns are lab IDs, lab names, and scientist
%    IDs. Write an SQL statement that outputs the number of scientists
%    in each lab.
%    \begin{compactitem}
%    \item Could not complete.
%    \item Could complete with documentation or search engine help.
%    \item Could complete with little or no documentation or search engine help.
%    \end{compactitem}
%
%  \item
%    How familiar do you think you are with the command line?
%    \begin{compactitem}
%    \item Not familiar with the command line.
%    \item Only familiar with the term.
%    \item Familiar with the command line but have never used it.
%    \item Familiar with the command line because I have or am using it.
%    \end{compactitem}

  \item
    How would you solve this problem: A directory contains 1000 text
    files. Create a list of all files that contain the word
    ``Drosophila'' and save the result to a file called results.txt.
    \begin{compactitem}
    \item Could not create this list.
    \item Would create this list using ``Find in Files'' and ``copy and paste''.
    \item Would create this list using basic command line programs.
    \item Would create this list using a pipeline of command line programs.
    \end{compactitem}

\end{itemize}

\subsection{Survey Questions}
\label{sec:survey}

The following is a subset of questions from the survey conducted by 
Dr. Aranda: 

\begin{compactitem}
\item In a typical week, about how many hours do you work? 
(If you are a student, your studies count as work.)

\item About how many of those hours do you spend creating, modifying, or testing software?

\item Indicate your level of use of each of the following tools or techniques: [Likert scale with three values (Do Not Use, Sometimes, Frequently) plus a No Answer option.]
  \begin{compactitem}
  \item Shell commands
  \item Testing
  \item SQL
  \item Version control
  \item Python
  \end{compactitem}
  
\item Do you understand the following shell commands well enough to explain them to somebody else? [Options: Yes, No, No answer.]
  \begin{compactitem}
  \item ls data/*.txt
  \item sort elements.txt $>$ elements.txt
  \item find $\sim$-name `*.py'
  \item ps -A $|$ grep mysample
  \end{compactitem}

\item Do you have research goals that you cannot attain because of a lack of computational or programming expertise? If so, please elaborate.

\item Can you think of computer-related things that you are doing differently than you used to since taking the workshop? If so, please elaborate.
\end{compactitem}

%---
The following questions illustrate the type of analysis performed by
Prof.\ Libarkin: 

\begin{compactitem}
\item Participants were asked to indicate the extent to which they understood specific concepts, with ratings of Strongly Disagree, Disagree, Agree, and Strongly Agree. A score of 1 implies low understanding (Strong Disagreement), while a score of 4 implies high understanding (Strong Agreement).

%Participants were asked to indicate the extent to which they
%agreed with statements about their Python coding ability, with ratings of Strongly Disagree,
%Disagree, Agree, and Strongly Agree. A score of 1 implies low ability (Strong Disagreement),
%while a score of 4 implies high ability (Strong Agreement).

\item Five questions asked participants to rate components of the
workshop, as well as the overall workshop on a 5-point Likert scale of Very Poor-Poor-Adequate-Good-Very Good. %A 1 corresponds to a Very Poor rating and a 5 corresponds to a
%Very Good Rating. Participants were also asked if the workshop met their needs on a 4-point scale (4=very well), if they learned as expected from the workshop, if understanding of computational science changed, and if they would recommend the workshop.

\item Participants responded to the prompt: ``Please provide any additional comments about your expectations for the workshop below."
\end{compactitem}

\subsection{Interview Questions}
\label{sec:survey}

The following is a subset of questions from a pre-workshop interview 
script used in Dr.\ Aranda's assessment research:

\begin{compactitem}
\item Do you use the shell? How comfortable are you using it? Examples?

\item Do you use Python? How comfortable are you using it? Examples? How about other programming languages? How did you learn to use them?

\item How do you manage the versions of your code and data? Do you use version control software? For what purpose? How?

\item What can you tell me about your testing practices? Do you have automated tests? How do you assess if your software behaves appropriately?

\item Are you concerned about data or code provenance issues? What do you do about it?
\end{compactitem}

The post-workshop interview script asked the same questions and added
some additional ones, like:

\begin{compactitem}
\item Has anything changed in your approach to computing or software development since you took the workshop? Please walk me through an example.

\item Have your routines changed? In what ways?

\item Did any of the more philosophical or strategic points that the instructor made stick with you? If so, can you give me some examples?

\item Are you tackling the same research questions? Have they evolved?
\end{compactitem}

%%  END PROJECT NARRATIVE   (15 page limit)
% --------------- Excluded from page limitations
\appendix

\newpage
\pagenumbering{arabic}
\renewcommand{\thepage} {\footnotesize References\,---\,\arabic{page}}

% ------------------------------------------------------------------- References

\bibliographystyle{plain}
\bibliography{./proposal}

\end{document}
